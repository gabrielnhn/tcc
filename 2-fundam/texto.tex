\chapter{Related Work}


\section{Driver Monitoring}

The implementation of image-based driver attention monitoring systems can be done by analyzing data such as direction of gaze, head pose, fatigue/drowsiness detection, disposition of the hands, and more \cite{jOrtega2020}.

For instance, Yang et al. \cite{9053659} used head pose, gaze direction, yawning and eye state analysis such as blink rate, blink duration and eye open/close in a single CNN framework; using multi-tasking and bringing together all the data to determine how distracted and drowsy the driver is.

Ortegaet al. \cite{eyelid_aperture} also proposed that calculating the eyelid aperture of drivers could be used to detect microsleeping blink patterns and more fatigue levels.

Attention-detecting systems could also use full-body images to detect distractions such as grabbing items like water bottles or telephones, and interacting with other passengers \cite{jOrtega2020}; and a camera pointed to the driver's hands could also tell whether the driver has his attention on the road.

\section{Appearance-based Gaze Estimation}

The task of detecting gaze direction in unconstrained environments successfully is a novelty. Previously, images used to train gaze estimation models were collected under controlled laboratory conditions; It was not until 2015 until Zhang et al.\cite{7299081} published the MPIIGaze dataset, with more than 210,000 images, taken in a span of three months, with variable lightning and background conditions.



\section{Conclusão}

Todo capítulo (com exceção da introdução e da conclusão) deve encerrar com uma pequena conclusão local, resumindo os tópicos apresentados no capítulo e preparando o leitor para o próximo capítulo (exceto se esse for a conclusão geral). Caso o capítulo tenha apresentado resultados obtidos pelo próprio autor, estes devem ser sucintamente relembrados aqui.

%=====================================================
